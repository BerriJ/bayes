\documentclass[12pt,a4paper]{article}
\usepackage{lmodern}

\usepackage{amssymb,amsmath}
\usepackage{ifxetex,ifluatex}
\usepackage{fixltx2e} % provides \textsubscript
\ifnum 0\ifxetex 1\fi\ifluatex 1\fi=0 % if pdftex
  \usepackage[T1]{fontenc}
  \usepackage[utf8]{inputenc}
\else % if luatex or xelatex
  \ifxetex
    \usepackage{mathspec}
    \usepackage{xltxtra,xunicode}
  \else
    \usepackage{fontspec}
  \fi
  \defaultfontfeatures{Mapping=tex-text,Scale=MatchLowercase}
  \newcommand{\euro}{€}
\fi
% use upquote if available, for straight quotes in verbatim environments
\IfFileExists{upquote.sty}{\usepackage{upquote}}{}
% use microtype if available
\IfFileExists{microtype.sty}{%
\usepackage{microtype}
\UseMicrotypeSet[protrusion]{basicmath} % disable protrusion for tt fonts
}{}
\usepackage[lmargin = 5cm,rmargin = 2.5cm,tmargin = 2.5cm,bmargin = 2.5cm]{geometry}

% Figure Placement:
\usepackage{float}
\let\origfigure\figure
\let\endorigfigure\endfigure
\renewenvironment{figure}[1][2] {
    \expandafter\origfigure\expandafter[H]
} {
    \endorigfigure
}

%%%% Jens %%%%
\DeclareMathOperator*{\argmax}{arg\,max}
\DeclareMathOperator*{\argmin}{arg\,min}

\usepackage{numprint}
\npthousandsep{\,}

%% citation setup
\usepackage{csquotes}

\usepackage[backend=biber, maxbibnames = 99, style = apa]{biblatex}
\setlength\bibitemsep{1.5\itemsep}
\addbibresource{R_packages.bib}
\bibliography{references.bib}
\usepackage{longtable,booktabs}
\usepackage{graphicx}
\makeatletter
\def\maxwidth{\ifdim\Gin@nat@width>\linewidth\linewidth\else\Gin@nat@width\fi}
\def\maxheight{\ifdim\Gin@nat@height>\textheight\textheight\else\Gin@nat@height\fi}
\makeatother
% Scale images if necessary, so that they will not overflow the page
% margins by default, and it is still possible to overwrite the defaults
% using explicit options in \includegraphics[width, height, ...]{}
\setkeys{Gin}{width=\maxwidth,height=\maxheight,keepaspectratio}
\ifxetex
  \usepackage[setpagesize=false, % page size defined by xetex
              unicode=false, % unicode breaks when used with xetex
              xetex]{hyperref}
\else
  \usepackage[unicode=true, linktocpage = TRUE]{hyperref}
\fi
\hypersetup{breaklinks=true,
            bookmarks=true,
            pdfauthor={Jens Klenke},
            pdftitle={Bayes Seminar},
            colorlinks=true,
            citecolor=black,
            urlcolor=black,
            linkcolor=black,
            pdfborder={0 0 0}}
\urlstyle{same}  % don't use monospace font for urls
\setlength{\parindent}{0pt}
\setlength{\parskip}{6pt plus 2pt minus 1pt}
\setlength{\emergencystretch}{3em}  % prevent overfull lines
\setcounter{secnumdepth}{5}

%%% Use protect on footnotes to avoid problems with footnotes in titles
\let\rmarkdownfootnote\footnote%
\def\footnote{\protect\rmarkdownfootnote}

%%% Change title format to be more compact
\usepackage{titling}

% Create subtitle command for use in maketitle
\newcommand{\subtitle}[1]{
  \posttitle{
    \begin{center}\large#1\end{center}
    }
}

\setlength{\droptitle}{-2em}
  \title{Bayes Seminar}
  \pretitle{\vspace{\droptitle}\centering\huge}
  \posttitle{\par}
\subtitle{Advanced R for Econometricians}
  \author{Jens Klenke}
  \preauthor{\centering\large\emph}
  \postauthor{\par}
  \predate{\centering\large\emph}
  \postdate{\par}
  \date{today}


%% linespread settings

\usepackage{setspace}

\onehalfspacing

% Language Setup

\usepackage{ifthen}
\usepackage{iflang}
\usepackage[super]{nth}
\usepackage[ngerman, english]{babel}

%Acronyms
\usepackage[printonlyused, withpage, nohyperlinks]{acronym}
\usepackage{changepage}

% Multicols for the Title page
\usepackage{multicol}

\begin{document}

\selectlanguage{english}

%%%%%%%%%%%%%% Jens %%%%%
\numberwithin{equation}{section}


%\maketitle

\begin{titlepage}
  \noindent\begin{minipage}{0.6\textwidth}
	  \IfLanguageName{english}{University of Duisburg-Essen}{Universität Duisburg-Essen}\\
	  \IfLanguageName{english}{Faculty of Business Administration and Economics}{Fakultät für Wirtschaftswissensschaften}\\
	  \IfLanguageName{english}{Chair of Econometrics}{Lehrstuhl für Ökonometrie}\\
  \end{minipage}
	\begin{minipage}{0.4\textwidth}
	  \begin{flushright}
  	  \vspace{-0.5cm}
      \IfLanguageName{english}{\includegraphics*[width=5cm]{Includes/duelogo_en.png}}{\includegraphics*[width=5cm]{Includes/duelogo_de.png}}
	  \end{flushright}
	\end{minipage}
  \\
  \vspace{1.5cm}
  \begin{center}
  \huge{Bayes Seminar}\\
  \vspace{.25cm}
  \Large{Advanced R for Econometricians}\\
  \vspace{0.5cm}
  \large{Seminar Paper}\\
  \vspace{1cm}
  \large{  \IfLanguageName{english}{Submitted to the Faculty of \\ Ökonometrie  \\at the \\University of Duisburg-Essen}{Vorgelegt der \\Fakultät für Wirtschaftswissenschaften der \\ Universität Duisburg-Essen}\\}
  \vspace{0.75cm}
  \large{\IfLanguageName{english}{from:}{von:}}\\
  \vspace{0.5cm}
  Jens Klenke\\
  \end{center}
  %\vspace{2cm}
  \vfill
  \hrulefill

  \noindent\begin{minipage}[t]{0.3\textwidth}
  \IfLanguageName{english}{Reviewer:}{Erstgutachter:}
  \end{minipage}
  \begin{minipage}[t]{0.7\textwidth}
  \hspace{1cm}Christoph Hanck
  \end{minipage}

  \noindent\begin{minipage}[t]{0.3\textwidth}
  \IfLanguageName{english}{Deadline:}{Abgabefrist:}
  \end{minipage}
  \begin{minipage}[t]{0.7\textwidth}
  \hspace{1cm}Jan.~17th 2020
  \end{minipage}

  \hrulefill

  \begin{multicols}{2}

  Name:

  Matriculation Number:

  E-Mail:

  Study Path:

  Semester:

  Graduation (est.):
 
  \columnbreak

  Jens Klenke

  3071594
  
  jens.klenke@stud.uni-due.de

  M.Sc. Economics

  \nth{5}

  Winter Term 2020

	\end{multicols}

\end{titlepage}

\newgeometry{top=2cm, left = 5cm, right = 2.5cm, bottom = 2.5cm}


\pagenumbering{Roman}
{
\hypersetup{linkcolor=black}

\setcounter{tocdepth}{3}
\tableofcontents
}

\newpage
\listoffigures
\addcontentsline{toc}{section}{List of Figures}

%\newpage
\listoftables
\addcontentsline{toc}{section}{List of Tables}

\section*{List of Abbreviations}
\addcontentsline{toc}{section}{List of Abbreviations}

\begin{adjustwidth}{1.5em}{0pt}

\begin{acronym}[dummyyyy]
 \acro{bagging}{Bootstrap Aggregation}
 \acro{LASSO}{Least Absolute Shrinkage and Selection Operator}
 \acro{pcr}{Principal Components Regression}
 \acro{pls}{Partial Least Squares}
 \acro{RMSE}{Root Mean Squared Error}
 \acro{MCMC}{Markov chain Monte Carlo} 
 \acroplural{LRG}[LRG]{längefristige Refinanzierungsgeschäfte}

%Falls eine Abkürzung in der Mehrzahl nicht einfach auf "s" endet muss das speziell eingestellt werden.
% \acro{slmtA}{super lange mega tolle Abkürzung} %Einzahl
 %\acroplural{slmtA}[slmtAs]{super lange mega tolle Abkürzungen} %Mehrzahl
 \acro{dummyyyy}{dummyyy}
\end{acronym}

\end{adjustwidth}

\restoregeometry

\newpage
\pagenumbering{arabic}
\hypertarget{introduction}{%
\section{Introduction}\label{introduction}}

In recent years, the \ac{LASSO} method of TIBSHIRANI has emerged as an
alternative to ordinary least squares estimation. The success of the
method is mainly due to its ability to perform both variable selection
and estimation. As already Tibshirani pointed out in his original paper
the standard \ac{LASSO} model can be interpreted as a linear regression
with a Laplace prior. PARK and CASELLA where the first to implement the
Bayesian l\ac{LASSO} \textgreater\textgreater using a conditional
Laplace prior specification\textless\textless.

Our goal is to compare the result of the Bayesian \ac{LASSO} with normal
\ac{LASSO} method and an ordinary least square estimation. The focus is
particularly on the number of non-significant parameters in the linear
model or, in case of the \acp{LASSO} the parameters equal to zero.

\newpage

\hypertarget{theory-of-bayesian-inference}{%
\section{Theory of Bayesian
inference}\label{theory-of-bayesian-inference}}

The Bayesian (inference) statistics based on the Bayes' theorem for
events.

\begin{align}
\label{eq:bayes_theorem}
  P(A | B) = \dfrac{P (B | A) P(A)}{P(B)}
\end{align}

For Bayesian statistics the event theorem gets \eqref{eq:bayes_theorem}
rewritten to apply it to densities. Where \(\pi (\theta)\) is the prior
distribution - which could be gained from prior research or knowledge,
\(f(y | \theta )\) is the likelihood function, and \(\pi (\theta| y)\)
is the posterior distribution, we then get the following.

\begin{align}
\label{eq:bayes_dens}
  \pi (\theta | y) = \dfrac{f(y | \theta) \pi(\theta)}{f(y)}
\end{align}

From \eqref{eq:bayes_dens} the advantages and disadvantages of Bayesian
statistics compared to frequentist statistics can directly be retrieved.
One major adavantage is that the Bayesian approach can account for prior
knowledge and points out a philosophical difference to the frequentist
approach - that the obtained data stands not alone. Another, key
difference and advantage is that in the Bayesian world the computation
are made with distributions and this leads to a better information level
than just the computation of the first and second moment. The
computation of distributions are also the greatest disadvantages or more
neutral the biggest problem of the Bayesian approach because in high
dimensional problems the computation takes a lot of times or is
sometimes even not possible. A solution to that is that with newer and
better computers it is possible to simulate the integrals with a
\ac{MCMC} method. GHOSH

\newpage

\hypertarget{data-description}{%
\section{Data description}\label{data-description}}

We collected the data from the online database platform \emph{kaggel.}
ZITATE The dataset included 6 years of data for all players which where
included in the soccer simulation game \emph{FIFA} from \emph{EA
Sports}. We dicided to just keep the data for 2019 and 2020. The Data
for 2019 will be used for the estimation of the differen models whereas
the 2020 data will be used to compare quality of the model with an out
of sample \ac{RMSE}.

\begin{longtable}[]{@{}lllrlrr@{}}
\caption{\label{tab:sum} Summary}\tabularnewline
\toprule
& year & & N & & mean & sd\tabularnewline
\midrule
\endfirsthead
\toprule
& year & & N & & mean & sd\tabularnewline
\midrule
\endhead
value\_eur & 2019 & & 17538 & & 2473043.68 & 5674963.22\tabularnewline
& 2020 & & 18028 & & 2518484.58 & 5616359.21\tabularnewline
wage\_eur & 2019 & & 17538 & & 10085.87 & 22448.70\tabularnewline
& 2020 & & 18028 & & 9584.81 & 21470.29\tabularnewline
overall & 2019 & & 17538 & & 66.23 & 7.01\tabularnewline
& 2020 & & 18028 & & 66.21 & 6.95\tabularnewline
\bottomrule
\end{longtable}

As one can see in Table \ref{tab:sum} the differences between the
versions for the most important variables are considerable small. For
the variable overall NA

\newpage

\hypertarget{methodic-procedure}{%
\section{Methodic Procedure}\label{methodic-procedure}}

To compare the Bayesian \ac{LASSO} we will analyse the data also with a
linear multivariate model, and the frequentist \ac{LASSO}. We wil start
with the linear model and will modifie the model equations step by step
forward the bayesian version.

\hypertarget{linear-model}{%
\subsection{Linear Model}\label{linear-model}}

The frequentist multivariate regression model has the follwing model
equation.

\begin{align}
\label{eq:lm}
\pmb{Y = \beta_0 + X \beta} + \pmb{\epsilon}
\end{align}

Where the coefficient will be estimated by the ordinary least square
method, which means that \(\pmb{\beta}\) should be chosen so that the
\emph{Euclidean norm} \(\left( || \mathbf{y - X\beta} ||_2 \right)\) is
minimal. This yields in the conditon for the estimation of coefficients:

\begin{align}
\label{eq:lm_con}
 \hat{\pmb{\beta}} = \argmin_{ \pmb{\beta}} (\pmb{y - \beta_0 - X  \beta})^T (\pmb{y - \beta_0 - X  \beta})
\end{align}

\hypertarget{section}{%
\subsection{\texorpdfstring{\acf{LASSO}}{}}\label{section}}

In the \ac{LASSO} method the model equation is the same as the equation
for the multivariate but the condition for the optimization of the
estimators has an additional punishment term.

\begin{align}
\label{eq:la_con}
\hat{\pmb{\beta}} = \argmin_{\pmb{\beta}}  \left( \pmb{y - X \beta} \right)^T \left( \pmb{y - X \beta} \right) + \lambda \sum_{i = 1}^{p} |\beta_j|
\end{align}

\hypertarget{bayesian-lasso}{%
\subsection{Bayesian Lasso}\label{bayesian-lasso}}

\hypertarget{gibbs-sampler}{%
\subsubsection{Gibbs Sampler}\label{gibbs-sampler}}

\hypertarget{estimation-of-the-bayesian-lasso}{%
\section{Estimation of the Bayesian
Lasso}\label{estimation-of-the-bayesian-lasso}}

\hypertarget{posteriod-based-estimation-and-prediction}{%
\section{Posteriod-based Estimation and
prediction}\label{posteriod-based-estimation-and-prediction}}

\hypertarget{residuals-and-sensitive-analysis}{%
\section{Residuals and Sensitive
Analysis}\label{residuals-and-sensitive-analysis}}

\hypertarget{discussion-and-further-research}{%
\section{Discussion and further
research}\label{discussion-and-further-research}}

\hypertarget{references}{%
\section{References}\label{references}}

\begin{align*}
  \quad - 
\end{align*}

\newpage
\textbf{Eidesstattliche Versicherung}

\bigskip

Ich versichere an Eides statt durch meine Unterschrift, dass ich die vorstehende Arbeit selbständig und ohne fremde Hilfe angefertigt und alle Stellen, die ich wörtlich oder annähernd wörtlich aus Veröffentlichungen entnommen habe, als solche kenntlich gemacht habe, mich auch keiner anderen als der angegebenen Literatur oder sonstiger Hilfsmittel bedient habe. Die Arbeit hat in dieser oder ähnlicher Form noch keiner anderen Prüfungsbehörde vorgelegen.

\vspace{1cm}
\rule{0pt}{2\baselineskip} %
\par\noindent\makebox[2.25in]{\indent Essen, den \hrulefill} \hfill\makebox[2.25in]{\hrulefill}%
\par\noindent\makebox[2.25in][l]{} \hfill\makebox[2.25in][c]{Jens Klenke}%


\end{document}
